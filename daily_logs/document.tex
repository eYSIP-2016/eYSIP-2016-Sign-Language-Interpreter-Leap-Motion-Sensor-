\documentclass[12pt,a4paper,english]{article}
\usepackage{graphicx}
\topmargin -0.6in
\oddsidemargin=0.0in
\evensidemargin=0.0in
\marginparwidth=0.6in
\textwidth=6.6in
\textheight=9.8in
\begin{document}
		\begin{flushleft}
				\Huge{ \textbf{Day 1:}}
		\end{flushleft}
		\begin{itemize}
			\item learn basic of python
			\item reference:
			 \begin{itemize}
				\item https://trutorialpoint.com
				\item https://stackoverflow.com
				\item https://youtube.com
			\end{itemize}
			\item getting started with leap motion sensor
			\begin{itemize}
				\item https://leapmotion.com/setup
				\item https://developer.leapmotion.com/documentation/python
			\end{itemize}
		\end{itemize}
		\begin{flushleft}
		\Huge{ \textbf{Day 2:}}
		\end{flushleft}
		\begin{itemize}
			\item starting programming with leap motion sensor which detect alphabet from figure position and direction
			\item I have used two factor figures tip position and position of center of palm
			(tip position – palm center position)
			\item So I could get the position of tip with respect to palm center so it doesn’t effect the change in position of hand with respect to sensor
			\item I take 300 reading and make its average.
			\item Here I got one problem which I couldn’t solve that day.
			\item The problem is like,
			\begin{itemize}
				\item I was storing data to object. But I am new to python so by mistake I was just passing the pointer the object it not copied but only pointer was passing
				\item So whenever the data is changed, that reflected at all object. I was confused why that was happening.
			\end{itemize}
		\end{itemize}
		\begin{flushleft}
		\Huge{ \textbf{Day 3:}}
		\end{flushleft}
		\begin{itemize}
			\item Today I solved that object coping problem. By this way\newline
		Initially:\newline	letter.fing1 = SimpleListener.fr1\newline
		letter.fing2 = SimpleListener.fr2\newline
		Finally:\newline letter.fing1 = copy.deepcopy(SimpleListener.fr1)\newline
		letter.fing2 = copy.deepcopy(SimpleListener.fr2)
		
			\item Then finally my program work. There were two option ‘l’ and ‘c’. ‘l’ means load and ‘c’ means check.
			\item So first load the posture of letter and then we can check
			\item and this is the readings \newline
			A\hspace{0.72in}		1	1	1	1	1	1	1	1\newline
			B\hspace{0.72in}		1	1	1	1	1	1	1	1\newline
			C\hspace{0.72in}		1	0	0	1	1	1	1	1\newline
			D\hspace{0.72in}		0	1	1	1	1	1	1	1\newline
			E\hspace{0.72in}		1	1	1	1	1	1	1	1\newline
			F\hspace{0.72in}		0	1	1	1	0	1	1	1\newline
			G\hspace{0.72in}		0	0	1	1	1	1	1	1\newline
			H\hspace{0.72in}		1	1	1	1	1	1	1	1\newline
			I\hspace{0.78in}		1	1	1	1	1	1	1	1\newline
			J\hspace{0.75in}		0	1	1	1	1	1	1	0\newline
			K\hspace{0.72in}		1	1	1	1	0	1	1	1\newline
			L\hspace{0.75in}		1	1	1	1	1	1	1	1\newline
			M\hspace{0.72in}		1	1	1	1	1	1	1	1\newline
			N\hspace{0.72in}		0	1	1	0	0	1	1	1\newline
			O\hspace{0.72in}		1	1	0	1	1	1	1	1\newline
			P\hspace{0.72in}		1	0	1	1	1	1	1	0\newline
			Q\hspace{0.72in}		1	1	1	1	1	0	1	1\newline
			R\hspace{0.72in}		0	1	0	1	0	0	0	1\newline
			S\hspace{0.72in}		1	1	1	1	1	1	1	1\newline
			T\hspace{0.72in}		1	1	1	1	1	1	1	1\newline
			U\hspace{0.72in}		1	0	1	1	1	1	1	1\newline
			V\hspace{0.72in}		1	1	1	1	1	1	1	1\newline
			W\hspace{0.66in}		1	1	1	1	1	1	1	1\newline
			X\hspace{0.72in}		1	1	1	1	1	1	1	1\newline
			Y\hspace{0.72in}		1	1	1	1	1	1	1	1\newline
			Z\hspace{0.75in}		0	1	1	1	1	1	1	1\newline
			
		
		\end{itemize}
\end{document}